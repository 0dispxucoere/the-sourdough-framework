In this chapter you will learn how to make your
own sourdough starter. Before doing so you will
quickly learn about baker's math. Don't worry,
it's a very simple way how to write recipe in
a cleaner more scalable way. Once you get the hang
of it you will want to write every recipe this way.
You will learn to understand the signs to determine
your starter's readiness.  Furthermore you will
also learn how to store your starter for
long-term storage.

\section{Baker's math}

In a large bakery a determining factor is how
much flour you have at hand. Based on the amount
of flour you have you can calculate how many
breads or buns you can make. To make it easy
for bakers the quantity of each ingredient
is calculated as a percentage based on how much flour you have.
Let me demonstrate this with a small example from
a pizzeria.  In the morning you check and you realize you
have around 1 kilogram of flour.
Your default recipe calls for around 600 grams of water.
That would be a typical pizza dough, not too dry but
also not too wet. Then you would be using around 20 grams
of salt and around 100 grams of sourdough starter.
\footnote{This is my go to pizza dough recipe. In Napoli
modern pizzerias would use fresh or dry yeast. However
traditionally pizza has always been made with sourdough.}
The next day you suddenly have 1.4 kilograms of flour
at hand and can thus make more pizza dough. What do you do?
Do you multiply all the ingredients by 1.4? Yes you could,
but there is an easier way. This is where baker's math
comes in handy. Let's look at the default recipe with baker's
math and then adjust it for the 1.4 kilogram flour quantity.

\begin{table}[H]
\centering
\resizebox{\textwidth}{!}{%
\begin{tabular}{|l|r|r|}
\hline
\textbf{Ingredient}    & \multicolumn{1}{l|}{\textbf{Explanation}} & \multicolumn{1}{l|}{\textbf{Explanation}}  \\ \hline
1000g flour            & 100\%                                     & 1000g of 1000g = 100\%                     \\ \hline
600g water             & 60\%                                      & 600g of 1000g = 60\%                       \\ \hline
100g sourdough starter & 10\%                                      & 100g of 1000g = 10\%                       \\ \hline
20g salt               & 2\%                                       & 20g of 1000g = 2\%                         \\ \hline
\end{tabular}%
}
\end{table}

Note how each of the ingredients is calculated as a percentage
based on the flour. The 100 percent is the baseline as the absolute
amount of flour that you have at hand. In this case that's 1000 grams
(1 kilogram).

Now let's go back to our example and just the flour as we have
more flour available the next day. As mentioned the next day
we have 1.4 kilograms at hand (1400 grams).

\begin{table}[H]
\centering
\resizebox{\textwidth}{!}{%
\begin{tabular}{|l|r|r|}
\hline
\textbf{Ingredient} & \multicolumn{1}{l|}{\textbf{Baker's math}} & \multicolumn{1}{l|}{\textbf{Calculated value}} \\ \hline
Flour               & 100\%                                      & 1400*1 = 1400g                                 \\ \hline
Water               & 60\%                                       & 1400*0.6 = 840g                                \\ \hline
Sourdough starter   & 10\%                                       & 1400*0.1 = 140g                                \\ \hline
Salt                & 2\%                                        & 1400*0.02 = 28g                                \\ \hline
\end{tabular}%
}
\end{table}

For each ingredient we calculate the percentage 
based on the flour available (1400 grams.) So for the water
we calculate 60 percent based on 1400. Open up your
calculator and type in 1400 * 0.6 and you have
the absolute value in grams that you should be using.
In that case that is 840 grams. Proceed and do the same
thing for all the other ingredients and you know
your recipe. 

Let's say you would want to use 50 kilograms of flour 
the next day. What would you do? You would simply proceed
and calculate the percentages one more time. I like this
way of writing recipes a lot. Imagine you wanted to make
some pasta. You would like to know how much sauce you should
be making. Now rather than making a recipe just for you, the
hungry family arrives. You are tasked with making pasta
for 20 people. How would you calculate the amount of sauce
you need? You go to the internet and check a recipe and then
are completely lost when trying to scale it up.


\section{The process of making a starter}
\section{How flour is fermented}
\section{Determining starter readiness}
\section{Maintenance}
\section{Longterm starter storage}