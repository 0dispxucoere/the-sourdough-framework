\documentclass{standalone}


\begin{document}
\begin{tikzpicture}[node distance = 3cm, auto]
  \node [block] (init) {\footnotesize Begin shaping};
  \node [decision, right of=init, node distance=5cm] (overfermented_decision) {\footnotesize Dough overly sticky or dough tears?};
  \node [block, right of=overfermented_decision, node distance=4cm] (overfermented) {\footnotesize Your dough is likely overfermented};
  \node [block, right of=overfermented, node distance=3cm] (loafpan) {\footnotesize Move to loaf pan, short proof, then bake directly};
  \node [block, below of=init, node distance=4cm] (shaping_technique) {\footnotesize Proceed with shaping technique};
  \node [block, right of=shaping_technique, node distance=3cm] (flour) {\footnotesize Flour shaped dough};
  \node [block, right of=flour, node distance=3cm] (banneton) {\footnotesize Place upside down in banneton};
  \node [block, right of=banneton, node distance=3cm] (proof) {\footnotesize Begin proofing};
  \path [line] (init) -- (overfermented_decision);
  \path [line] (overfermented_decision) -- node{yes} (overfermented);
  \path [line] (overfermented_decision) -- node{no} (shaping_technique);
  \path [line] (shaping_technique) -- (flour);
  \path [line] (flour) -- (banneton);
  \path [line] (banneton) -- (proof);
  \path [line] (overfermented) -- (loafpan);
\end{tikzpicture}
\end{document}
